\chapter{INTRODUÇÃO}
\label{intro}

\todo[inline]{
Este modelo é baseado na Classe Latex da FEI, uma ferramenta para padronização de textos acadêmicos do Centro Universitário FEI. Mais informações sobre a utilização da classe pode ser encontrado em sua página do Github (\url{https://github.com/douglasrizzo/Classe-Latex-FEI/blob/master/fei.pdf}).
O livro "Writing for publication" de \textcite{jalongo2016writing} foi utilizado para compor esse material, lá você pode encontrar mais informações sobre como escrever trabalhos acadêmicos. Utilize este modelo para construir o seu trabalho.
}

\todo[inline, color=blue!20!white]{
Durante o TCC, o aluno deve:
\\- Demonstrar sua capacidade de realizar uma pesquisa.
\\- Aplicar os conhecimentos adquiridos ao longo do curso.
\\- produzir um trabalho original e relevante na sua área de estudo.
}

\todo[inline]{Aqui temos a introdução. Comece apresentando os seguintes itens:
}


\todo[inline]{
\\- Contextualização:
Qual a importância do tema?  
Por que esse trabalho é interessante para o leitor?
Quais o principais conceitos?
Quais as principais aplicações?
\\- Motivação: 
Quais os fatores que determinam a escolha do tema? 
Qual a sua relação com o tema?
\\- Contribuição: 
Como seu trabalho contribui para a ciência?
Qual a sua contribuição para o avanço desse tema?
Qual a inovação que seu trabalho traz?
}


\section{Objetivo}
\todo[inline]{Quais as questões a serem respondidas? O Objetivo deve ser uma única frase. Deixe claro qual o problema que você pretende responder com a pesquisa, assim como sua delimitação espacial e temporal. Apresente os objetivos em termos claros e precisos, listando possíveis sub-problemas. Utilize verbos de ação como verificar, identificar, descrever e analisar. Se houver hipótese, deixar explicitas as relações da hipotes com as variáveis do problema}

\section{Estrutura do Trabalho}
\todo[inline]{O que o leitor irá encontrar nas próximas sessões}

O restante deste trabalho é dividido da seguinte maneira: 

No capítulo 2, serão apresentados todos os conceitos utilizados e relacionados ao tema abordado, para que o leitor possa entender com clareza as técnicas que estão sendo tratadas no trabalho e compreender os termos que serão descritos posteriormente.

No capítulo 3, serão descritos trabalhos relacionados disponíveis na literatura, com o objetivo de apresentar o cenário atual de pesquisa da área.
 
O capítulo 4 detalhará a metodologia que será utilizada para o desenvolvimento deste trabalho, demonstrando as técnicas que serão utilizadas e os passos a serem realizados para atingir o objetivo final.

O capítulo 5 irá expor o que os autores deste trabalho esperam ao longo do desenvolvimento e após a implementação da metodologia proposta.