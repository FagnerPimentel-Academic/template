\chapter{METODOLOGIA}
\todo[inline]{Nesta sessão devemos devemos deixar claro que tipo de pesquisa esta sendo realizada (Qualitativa/quantitativa) e como este trabalho pode ser reproduzido por outras pessoas. Para isso, após definir o tipo de pesquisa, especificamos três pontos: \\- Os matérias utilizados
\\- Os métodos aplicados
\\- As métricas avaliadas}

\section{Materiais}
\todo[inline]{Identifique suprimentos e equipamentos necessários para a realização da pesquisa. podendo ser hardware, software, base de dados, etc. Para cada material defina:
Qual o material de hardware e software utilizado? Quais as bases de dados já existentes que podem ser utilizadas e como elas foram selecionadas?
\\- O que é?
\\- Por que foi selecionado?
\\- Como funciona?
\\- Onde é aplicado?
}

\todo[inline]{Caso existam participantes voluntários na pesquisa, é preciso esclarecer os seguintes pontos para estes voluntários: 
\\- O Propósito da pesquisa. 
\\- Porquê e como eles foram selecionados. 
\\- Por quanto tempo eles participarão do experimento.
\\- Como os dados dos participante são manipulados em termos de confidencialidade e anonimato.
\\- Deixar claro que o participante é voluntário e não sofrerá nenhuma consequência negativa caso escolha não participar do experimento.
\\- Deixar claro que o participante é livre e não sofrerá nenhuma consequência negativa caso abandone o experimento em qualquer etapa.
}


\section{Métodos}

\todo[inline]{A descrição dos métodos utilizados deve ser simples e direta. Geralmente é descrito um pipeline ou um diagrama com etapas. Cada etapa explica uma fase do desenvolvimento do trabalho.
Em cada etapa, descreva:
\\- O Objetivo da etapa
\\- Como ela funciona
\\- Parâmetros e configurações utilizados
\\- Quais trabalhos relacionados utilizam os métodos apresnetados na etapa?
} 

\section{Métricas}
\todo[inline]{Você  descreve aqui como vai testar o seu trabalho. Pode ser feito de forma geral ou subdividido para cada etapa apresentada na sessão anterior. Para cada etapa, defina:
\\- O que será avaliado?
\\- Qual o objetivo desta avaliação?
\\- Quais dados serão coletados? 
\\- Como os dados serão coletados?
\\- Para quê serve esses dados?
\\- Como eles serão avaliados?
\\- Quantos experimentos serão utilizados?
\\- Quais trabalhos relacionados utilizam essas métricas?
}
